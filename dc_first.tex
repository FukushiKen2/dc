%\documentclass[11pt,a4paper,uplatex,dvipdfmx]{ujarticle} 		% for uplatex
\documentclass[11pt,a4j,dvipdfmx]{jarticle} 					% for platex
\input{pieces/form00_header} % pieces
\input{pieces/kakenhi7} % pieces
\input{pieces/form01_dcpd_header} % pieces
\input{pieces/hook3} % pieces
%#Name: dc
\input{pieces/form03_dcpd_headers} % pieces
\input{pieces/form04_dc_header} % pieces
% ===== Global definitions for the Kakenhi form ======================
% 基本情報
%
%------ 研究課題名  -------------------------------------------
\newcommand{\研究課題名}{象の卵}

%----- 研究機関名と研究代表者の氏名-----------------------
\newcommand{\研究機関名}{京都大学}
\newcommand{\研究代表者氏名}{福士謙二}
\newcommand{\me}{\underline{\underline{H.~Yukawa}}}
\input{pieces/inst_dcpd} % pieces
% user07_header
% ===== my favorite packages ====================================
% ここに、自分の使いたいパッケージを宣言して下さい。
\usepackage{wrapfig}
%\usepackage{amssymb}
%\usepackage{mb}
%\DeclareGraphicsRule{.tif}{png}{.png}{`convert #1 `dirname #1`/`basename #1 .tif`.png}
\usepackage{lineno}
\usepackage{amsmath}
\usepackage{amsfonts}
\usepackage{latexsym}
\usepackage{amsthm}
\usepackage{udline}
\usepackage[dvipdfmx]{graphicx}
\usepackage{here}


% ===== my personal definitions ==================================
% ここに、自分のよく使う記号などを定義して下さい。
\newcommand{\klpionn}{K_L \to \pi^0 \nu \overline{\nu}}
\newcommand{\kppipnn}{K^+ \to \pi^+ \nu \overline{\nu}}

% ----- 業績リスト用 -------------
\newcommand{\paper}[6]{%
	% paper{title}{authors}{journal}{vol}{pages}{year}
	\item ``#1'', #2, #3 {\bf #4}, #5 (#6).			% お好みに合わせて変えてください。
}

\newcommand{\etal}{\textit{et al.\ }}
\newcommand{\ca}[1]{*#1}	% corresponding author;   \ca{\yukawa}  みたいにして使う
\newcommand{\invitedtalk}{招待講演}

\newcommand{\yukawa}{H.~Yukawa}					% no underline
%\newcommand{\yukawa}{\underline{\underline{H.~Yukawa}}}	% with 2 underlines
\newcommand{\tomonaga}{S.~Tomonaga}

\newcommand{\prl}{Phys.\ Rev.\ Lett.\ }		% よく使う雑誌も定義すると楽

% ===== 欄外メモ ==================
\newcommand{\memo}[1]{\marginpar{#1}}
%\renewcommand{\memo}[1]{}	% 全てのメモを表示させないようにするには、行頭の"%"を消す

%\input{../../sample/simple/contents}	% skip
\input{pieces/hook5} % pieces

\begin{document}
\input{pieces/hook7} % pieces
%#Split: 01_background
%#PieceName: p01_background
\input{pieces/p01_background_00}
\section{研究の位置づけ}
%    <<最大 1ページ>>

%s03_background
%begin 本研究の着想に至った経緯など ====================
\noindent\Large
\textbf{研究テーマはSergei Novikovにより始められたclosed 1-formのゼロ点を調べるモース理論の一般化の理論の拡張である}

\normalsize
\vspace{2mm}
1. モース関数の代わりにclosed 1-formのゼロ点を扱うMorse Novikov理論の境界付き多様体への一般化

2. 特異点を持つ空間へのMorse Novikov理論を用いた新しいアプローチ

3. フレアーホモロジー等無限次元モース理論においてのMorse Novikov理論のアナロジーとなる汎関数  の開発


\vspace{2mm}
\noindent\Large
 \textbf{Morse Novikov理論とは}\normalsize
\vspace{1mm}

	多様体のトポロジーをモース関数を用いて調べるモース理論は古くからトポロジーにおいて最も重要な理論の一つであり,フレアーホモロジー等への一般化などその影響力は現在においても健在である.Novikovは\textbf{\ul{ゼロ点の近傍でモース関数の外微分と同じ形をしているclosed 1-form(以下Morse 1-formと呼ぶ)のゼロ点を調べること}}を提唱した.モース関数$f$に対し$df$のゼロ点と$f$の臨界点は一致するためNovikovの理論はモース理論の一般化になっている.\textbf{\ul{モース理論との一番の違いはgradient flowの挙動である}}.このflowの複雑さがより深い結果を導き出す.リーマン計量を通じて1-formとベクトル場のカノニカルな同型が存在するので(関数の勾配の一般化)そのベクトル場に対する負の方向のflowをgradient flowと呼ぶ.モース理論の場合flowに沿って関数の値は単調に減少する.Edward Wittenはモース関数の臨界点を生成元とするチェイン複体で,指数の差が1の臨界点を結ぶflowの本数を微分の係数とするホモロジー理論を構築した.closed 1-formの場合flowの挙動は簡単でなく,ホモロジー理論を作ろうにも臨界点を結ぶflowは無限に存在することがある.flowたちを基本群と整数環の群環を完備化したノビコフ環とよばれる形式的無限和に対応させ数え上げることでこの問題は解決されNovikov環係数のホモロジー理論が作られた.この手法はゲージ理論,シンプレクティック幾何での無限次元の状況下でflowを数え上げる際の下地になっている.このNovikov複体は強力な結果で,例えば3次元多様体上でCircle Valued Morse関数$(f:M \rightarrow S^1$のモース理論で,$[df]\in H^1(M,\mathbb{Z})$の時に対応)が作るNovikov複体のReidemeister torsionとゲージ理論のサイバーグウィッテン不変量が等価であるという予想がHutchingsにより提出された.flowに関連する別の話題としてLS categoryの一般化がある.それはコホモロジー$\xi \in H^1(M,\mathbb{R})$にゼロ以上の数$cat(M,\xi)$を定義するものである.古典的なLS categoryの有名な結果に閉多様体上の関数は$cat(M)$以上の臨界点を持つという結果があるが,Michael FarberはMorseでなく一般のclosed 1-form  $\omega$のゼロ点が$cat(M,[\omega])$未満ならば任意のflowはhomoclinic cycleを持つという力学系の非常に興味深い結果を証明した.申請者は\textbf{\ul{これらの結果を境界や特異点をもつ多様体やフレアーホモロジー理論へと拡張}}をするつもりである.次ページの研究内容の項で詳しく説明するが特に結び目補空間や曲面などの特異点をもつ4次元多様体のモノドロミー作用やゲージ理論などに対応する不変量が期待される.

\vspace{4mm}
\noindent\Large
	\textbf{着想の経緯}\normalsize
	\vspace{1mm}

	申請者は大学院で物理学科から数学科へ入学し入学後は微分幾何学やリーマン幾何学,モース理論を特に勉強し,その流れで指導教官に勧められ修士でMichael FarberのThe topology of closed 1-formを通読した.そしてこの理論はモース不等式,調和積分論,Reidemeister torsion,LS category等の華々しい幾何学の結果たちを精緻な結果で蘇らせ,さらに無限次元モース理論の土台となる意義深い理論であると確信し研究を決意するに至った.

%end 本研究の着想に至った経緯など ====================

\input{pieces/p01_background_01}

%#Split: 02_purpose_plan
%#PieceName: p02_purpose_plan
\input{pieces/p02_purpose_plan_00}
\section{研究目的・内容等}
%    <<最大 2ページ>>

%s02_purpose_plan_dcpd
%begin 研究目的と研究計画short留意事項なし ====================
\noindent
\fbox{\textbf{研究目的}}

Morse Novikov理論を境界や特異点をもつ多様体や無限次元モース理論に対して拡張し多様体のより精密な不変量を与えたり,多様体上の新たな解析のテクニックを開発することが研究の目的である.


\noindent
\fbox{\textbf{Morse Novikov理論のはじまりと今後}}

前項ではMorse Novikov理論を用いた成果について多く挙げたが,研究内容を述べる前にそもそもの理論の始まりと今後について述べたい.Novikovが始めたこの理論は\textbf{\ul{モース関数の多重値関数への一般化}}が念頭にあった.対数関数などの複素平面上の多価関数は被覆空間に引き戻すことで一価関数にすることができた.closed 1-formもしかるべき被覆空間へ引き戻せばexact formになることから,exactでないclosed 1-formを多価関数を外微分したような対象と捉える.複素関数の多価関数は経路を変えると値が変化する.closed 1-formを多価関数を微分したものと捉えるなら,ある自明でない基本群に対応するループに沿って積分することで値のズレが現れるはずで,実際コホモロジー類が0でないclosed 1-form $\omega$に対し積分が0にならないループが存在する.この洞察から被覆空間とモノドロミー作用の研究にMorse Novikov理論が適用できそうなことがわかる.Michael Farberによる教科書の出版によりモース理論のアナロジーとしての理論の構築がひと段落したところで,\textbf{\ul{今後はモノドロミー作用へ応用されていくべきである}}と申請者は考えている.



\noindent
\fbox{\textbf{研究内容・手法}}

以下前ページの研究テーマの項目に沿って研究内容および手法を詳しく述べる.

\vspace{1mm}
\noindent
\textbf{1. }
\textbf{\ul{境界付き多様体のモース理論はgradient flowが境界に対しtransverseかtangentであるかで大別される}}.境界に対しflowがtransverseな状況についての拡張は[1]でなされている.これに相当する結果がflowがtangentな状況下,\textbf{具体的には[3]の設定下}でも得られないか考えたい.[1]ではMorse Novikov理論を多様体$M$と境界上のflowが出ていく集合$B$に対し$(M,B)$の相対ホモロジーの結果に拡張している.\textbf{\ul{一番重要な事実は境界付き多様体においてもNovikov複体が構成されることであり,flowがtangentな場合に同じようにNovikov複体の構成}}を行いたい.

\vspace{1mm}
\noindent
\textbf{2.}
[2]ではtransverseな境界付き多様体のMorse Novikov理論においてモース不等式をWitten deformationと呼ばれる手法で証明している.Witten deformationとはモース関数$f$を用いて外微分を$e^{-tf}de^{tf}$に変形しラプラシアンの解をモース関数の臨界点に局所化しモース関数の臨界点とラプラシアンの解を対応付ける手法である.ホッジ理論よりラプラシアンの解はドラームコホモロジーと対応するためWitten deformationにより臨界点とベッチ数の対応,つまりモース不等式を解析的に導くことができる.[4]では特異点を持つ空間へのWitten deformationの拡張がされているが,\textbf{\ul{[2],[4]の結果をさらに特異点が次元を持つ状況,とりわけ結び目補空間や4次元多様体から曲面を除いた空間へ拡張する}}つもりである.
[4]は特異点を持つ多様体上のモース理論であるstratifiedモース理論を意識した拡張になっている.一般に特異点の周りのモノドロミーを考えることは非常に強力な手法で,Novikovの理論とモノドロミーの相性も良いため,Witten deformationの拡張に加え特異点をもつ多様体への新たなアプローチが得られると考えられる.さらにWitten deformationの顕著な応用にラプラシアンの固有値のゼータ関数から定義されるanalytic torsionとReidmeister torsionが等しくなるというCheeger-M\"{u}llerの定理の別証明がある.この結果の拡張にも取り組むつもりである.


\vspace{1mm}

\noindent
\textbf{3. }
多価関数に対してのモース理論というコンセプトはフレアーホモロジーへも大きな影響を与えている.フレアーホモロジーは数多くの種類があるが,モジュライなどの無限次元の対象に汎関数をあたえ,臨界点たちを生成元とするチェイン複体を考えチェインの微分をgradient flowの本数で与えるというアイデアは共通している.\textbf{\ul{[3]の境界付きモース理論の設定はラグランジアンフレアー理論が意識された設定であり,Novikov理論もその方向の拡張}}が考えられる.



\noindent
\fbox{\textbf{研究計画}}

各研究テーマの研究計画についてそれぞれ述べる.

\noindent
\textbf{1. }[1],[3]の論文は指導教官とのセミナーで読み終えたため,実際に\textbf{\ul{この論文たちのテクニックを用いてflowがtangentな境界付き多様体のNovikov複体を構成すること}}を試みたい.この方向性の研究は修士課程のこれからと博士課程初年度で行うつもりである.


\noindent
\textbf{2. }Witten deformationの拡張が今後取り組みたい課題であり,特異点が次元をもつ状況への拡張は今すぐ手をつけるべき課題と考えている.これはMorse Novikov理論の拡張に加え多様体上の解析のテクニックの新たな提案になると考えられる.博士課程初年度以降は
Cheeger-M\"{u}llerの定理よりNovikov複体におけるReidemeister torsionとAnalytic torsionの関係を考えたい.
Reidemeister torsionは被覆空間のモノドロミー作用と密接な関わりがあるため,前述の通りMorse Novikov理論と相性が良いと考えられる.実際結び目補空間でNovikov複体のReidemeister torsionを考えることは[5]などで考えられている.博士後期課程の初年度以降はこの論文を理解し,被覆空間へのモノドロミー作用に対しWitten deformationを用いた解析的なアプローチも試みるつもりである.また\textbf{\ul{HutchingによるReidemeister torsionとサイバーグウィッテン理論との関係の解析的なアプローチ又は後述のNovikov理論のフレアーホモロジーとの関係についての研究を博士後期課程の最終的な目標に据えることを考えている}}.



\noindent
\textbf{3. }フレアーホモロジーへの応用については博士後期課程2年目以降を考えており,前述のラグランジアンフレアーホモロジーへの拡張が考えられる.これは\textbf{1}の結果の延長になる.また結び目補空間で定義される種々のフレアー理論への拡張も考えられる.とりわけヒーガードフレアーホモロジーの多重値モース理論に対応したsuturedフレアーホモロジーへ応用できないか申請者は注目している.




\noindent
\fbox{\textbf{独創性}}

flowがtangentな境界付き多様体のモース理論のclosed 1-formへの一般化へ言及している論文やプレプリントを申請者が調べる限り確認することができなかった.そのためこのテーマはまだ誰にもなされていないと考える.さらにNovikov複体のtorsionとWitten deformationの関連についても調べるということは新しいアイデアであり,特異点とそのまわりのモノドロミーに対しての新しいアプローチになる.さらにこれらの有限次元の理論への深い洞察がフレアーホモロジーの新しい汎関数の発見につながると考えられる.

\vspace{2mm}
\noindent 参考文献

\noindent
[1] LI TIEQIANG,TAN. Topology of Closed 1-Forms on Manifolds with Boundary. Durham University Doctoral thesis, 2009.

\noindent
[2] Maxim Braverman and Valentin Silantyev. Kirwan-Novikov inequalities on a manifold with boundary. Transactions of the American Mathematical Society, 2004.

\noindent
[3] Akaho Manabu. MORSE HOMOLOGY AND MANIFOLDS WITH BOUNDARY. World Scientific, 2007.

\noindent
[4] Ursula Ludwig. An Extension of a Theorem by Cheeger and M\"{u}ller to Spaces with Isolated Conical Singularities. Comptes Rendus Mathematique, 2018.

\noindent
[5] Hiroshi GODA and Andrei V. PAJITNOV. Dynamics of gradient flows in the half-transversal Morse theory. Proceedings of the Japan Academy Series A Mathematical Sciences, 2009.


%end 研究目的と研究計画short留意事項なし ====================
\input{pieces/p02_purpose_plan_01}

%#Split: 03_rights
%#PieceName: p03_rights
\input{pieces/p03_rights_00}
\section{人権の保護及び法令等の遵守への対応}
%    <<最大 1ページ>>

% s09_rights
%begin 人権の保護及び法令等の遵守への対応 ====================
手続きが必要な研究は行わない予定である.

%end 人権の保護及び法令等の遵守への対応 ====================

\input{pieces/p03_rights_01}

%#Split: 04_abilities
%#PieceName: p04_abilities
\input{pieces/p04_abilities_00}
\section{研究遂行力の自己分析}
%    <<最大 2ページ>>

% s14_abilities
%begin 自己分析 ====================
%\DCPDInstructionsA\\% <-- 留意事項:これは消すか、コメントアウトしてください。

\noindent
\textbf{(1) 研究に関する自身の強み}
%\DCPDInstructionsB% <-- 留意事項:これは消すか、コメントアウトしてください。

申請者は
申請者は研究を遂行するにあたってのさまざまな強みを有している.一番の強みは幅広い知識量や積極性であり,それは申請者が学部では物理学を,そして修士では数学を専攻し幅広い知識を有していることから言える.
その知識量から複眼的な考察が可能であり数学や物理学のさまざまな分野の架け橋になる研究者としての資質を有している.以下に自身の強みに対して詳細を述べたい.



\noindent
\fbox{\textbf{主体性}}

申請者は京都大学の物理学科を卒業したのち修士課程で同大学の数学科へ入学した.それは物理学の定式化に用いられる洗練された数学の理論に魅力を覚えたからである.物理学の講義を学部のうちに一通り履修し,大学の数学も勉強はしていたものの数学科の講義全てに出席していたわけでなかった.そのため図書館で必死に大学院入試の勉強をこなし合格し,その後も筆記試験と面接を経て大学内の博士後期課程へ進学するコースに認められることができた.これは申請者が幅広い関心をもち自ら動く能力を有していることを表している.そのほか大学内の計算機の管理のオフィスアシスタントの活動,ティーチングアシスタントという学部生の微分積分学や線形代数学の課題の採点の仕事を積極的にこなした.これらの経験から申請者は主体的に動く力を十分に持っていると考えられる.


\noindent
\fbox{\textbf{知識量}}


\begin{wrapfigure}{r}{10zw}
    \vspace*{-\intextsep} %
	\includegraphics[width=20mm,scale=10]{figure0.pdf}

\end{wrapfigure}



申請者は学部時代より数学や物理学の学習を積極的に行ってきた.物理学科のころは電磁気学,統計力学,量子力学に加え一般相対性理論をよく学んだ.研究室は宇宙物理学の研究室で前期は一般相対性理論のテキストの輪読,後期は惑星と衛星が潮汐力により自転の周期が同期する現象の数値計算の卒業研究を行った.大学院からは数学科へ移り普段のセミナーをこなしつつ微分幾何学,調和積分論,特性類,ゲージ理論の知識をキャッチアップした.物理学科のころから現在に至るまでの学習の過程はつながっている.まず物理学の学習を通して,物理学を記述する言語としての多様体論,微分幾何学,ゲージ理論に関心を抱いた.現代の物理学は一般相対性理論と量子力学が2つの柱であり,それらの理論の定式化にはそれぞれ深い数学が用いられる.相対性理論は光速度不変という原則を空間と時間を入れた計量の入った4次元の多様体の幾何学で定式化することで確立された.
また現代の力学はラグランジアンとよばれる汎関数を変分して得られる方程式が物理現象を表すという最小作用の原理に基づいた理論構築がなされており,量子力学もそのアナロジーで定式化が進んでいる.そしてゲージ群と呼ばれる対称性を課したラグランジアンを扱うという物理学におけるゲージ理論は,ゲージ群$G$をファイバーにもつ主$G$束を扱う理論として数学的に定式化された.申請者はこのような物理学を数学的に明晰に扱う様を魅力に感じた.数学としてのゲージ理論の花形であるドナルドソン理論とは主$G$束の共変微分を表す接続と呼ばれる対象全体の空間に不変量を見出すという,ベクトル束の不変量を扱う特性類の考えを発展させた理論である.インスタントンフレアーホモロジーはさらに調和積分などの解析の手法を用いて接続全体の空間に汎関数を定義し無限次元モース理論を遂行させて定義される.フレアーホモロジー理論の構築にはモース理論のさまざまな技術が応用されておりMorse Novikov理論も例外ではない.このように申請者の過去の学習は現在の数学の最先端の理論までつながっている.
さらにNovikovの多重値のモース理論というアイデアも数理物理,ハミルトン力学系のフローのゼロ点の数の評価などから生まれたアイデアであり,Witten deformationも場の理論における超対称性とよばれる理論とモース理論との関係を追求する中で生まれたものである.このように申請者の本研究活動は数学と物理学に範囲が広く及ぶ遠大なテーマである.そのため数学科の学生に比べ幅広く学習を行い,新たな視点を提供できるであろう申請者は本研究活動において格好の人物である.\textbf{\ul{特にNovikovの元々の数理物理へのアプローチは現在のMorse Novikov理論の主流ではないように感じるため,本研究活動のほかにオリジナリティのある課題を見つけることができる可能性が高いだろう}}.


\noindent
\fbox{\textbf{セミナーの主催者としての能力}}

申請者は京都大学内でオフィスアシスタントと呼ばれる計算機の管理の仕事を修士から行っている.プログラミングにもとより関心が強く学習もしておりそのような仕事に関心があった.主な仕事内容は数学科のwebページの管理をはじめ,PCやプリンタ等の大学内で使用する機器の使用方法のマニュアルの作成などである.大学内のサーバーの構造や通信の仕組みについて日々の業務によりその知識を深めている.新型コロナウイルスはセミナーや研究集会の在り方を大幅に変化させ,オンライン化の指向性を極めて高めた.今後もこの流れは変わらないと考えられるため,オフィスアシスタントで得た知識や経験は貴重なものであり,オンラインで開催されたセミナーでのトラブルの対応など研究集会の主催者として適切な能力を有していると考えられる.



\noindent
\fbox{\textbf{向上心}}


申請者は大学院で物理学から数学へ専攻を変えたように,極めて強い好奇心や向上心,粘り強さで数学,物理学,プログラミングの学習や研究活動に取り組んできた.この飽くなき探究心でさらなる分野の発展に貢献したい.学部生の頃の実験の演習では長らく器具のずれにより実験結果がずれる現象があったがそれを補正するモデルを創出することを,数学力を持っていることから班員から任せられたように広い知識を持つがゆえに頼れることを経験してきた.数学の分野においてもこの問題の解決力,創造力を活かしたい.

学習面においては\textbf{\ul{大学院の講義で優秀な成績を収め授業料は全て免除され,京都大学内の理学研究科数学・数理解析専攻学生に対する支援に採用され支援金が給付された}}.これは申請者が大学院で優秀な成績を修めてきたことを表している.
普段の講義や研究室内のセミナーに加え,大学内の微分トポロジーセミナーや関西ゲージ理論セミナー,外部の研究集会では微分トポロジー'21,名古屋大学で主催された「4次元多様体I,II」の輪読会等多くの研究集会へ積極的に参加してきた.これらの知識は今後の研究者としての活動に大切なものでありこれからも積極的に研究集会へ参加をしていきたい.





\vspace{5mm}
\noindent
\textbf{(2) 今後研究者として更なる発展のため必要と考えている要素}


研究者としての更なる発展には数学のより深い知識,分野を横断するような幅広い見識が必要である.そしてそれを可能にするのは互いに教え合う友人,自身の考えをわかりやすく伝える能力であると申請者は考える.


\noindent
\fbox{\textbf{知識や技能の拡充}}

論文やテキストを熟読することを継続していきたい.現在は自分で関心のある論文に目を通してそれを指導教官とのセミナーで発表をしているが,今後は上記のセミナー等の参加者と論文の輪読会を企画していくことで知識や知人を増やせると考えている.



\noindent
\fbox{\textbf{幅広い交友関係}}


異なる分野の知識が必要になるときなどに,互いに教え合ったり刺激しあう学友が必要なことがある.ゲージ理論では微分幾何学で定式化された公式を,実際に興味深い例に当てはめるときに代数幾何学などの知識が必要になることがある.また複素幾何学とシンプレクティック幾何学の双対性のミラー対称性や調和写像の分野では微分幾何学と代数幾何学の繋がりは昔から強く意識されている.そのような状況に対応するには幅広い知識,友人が必要である.特に申請者は大学院から数学を専攻しているため知り合いが多いわけでないため,不慣れな分野で迷い込まないよう的確なアドバイスをもらえる教えてくれる先達,同じ分野の志を共にする友人を多く持ちたい.セミナーや研究集会に積極的に参加し発言もしっかりしていくことで交友関係は広がっていく.


\noindent
\fbox{\textbf{伝達力の更なる向上}}


セミナーでのプレゼンテーションの能力の向上を目指したい.これは自らが学会やセミナーで発表することを重ねることで身についていくものである.セミナーやその後の議論で自分のアイデアを円滑に伝えられることは研究活動において非常に重要である.発表中の伝わりやすい言葉や話し方,スライドの作り方にこだわりながらセミナー発表をこなしていきたい.


%end 自己分析 ====================

\input{pieces/p04_abilities_01}

%#Split: 05_my_ambitions
%#PieceName: p05_my_ambitions
\input{pieces/p05_my_ambitions_00}
\section{目指す研究者像等}
%    <<最大 1ページ>>

% s17_my_ambitions
\noindent
\textbf{(1)目指す研究者像 {\footnotesize ※目指す研究者像に向けて身に付けるべき資質も含め記入してください。}}

%begin 目指す研究者像 ====================
申請者の理想の研究者像は「\textbf{幾何学およびその周辺分野の発展に貢献し続ける研究者}」である.自分が魅入られた数学,特に微分幾何学で最先端の研究を行いたい.それだけでなく常に全力で好奇心を保ち研究生活を駆け抜けたい.申請者はそのような研究者になるために以下の能力を兼ね備えた研究者になりたいと考える.


\noindent
\fbox{\textbf{1.深掘りする思考力}}

数学の最大の魅力は理論が抽象化されているが故の適応範囲の広さであると申請者は考える.数学が適応範囲が広い学問であるということは古くから科学の発展に数学が寄与してきたことが示している.抽象化されたインパクトの大きい理論を構築するためには現象の本質を抽出する能力が必要であり,透徹した理論に多く触れることで本質を見通す感覚を養われると申請者は考えている.


\noindent
\fbox{\textbf{2.問題を解決する能力}}

本質を抽出する能力が問題の解決の糸口をみつけたりあらたな理論を構築するアイデアを見つけてくれるならば,そのアイデアを遂行する能力も必要である.自身が数学に造詣が深いことや計算力を身に着けることももちろん重要だが,とりわけ他人と協力し問題を解決する能力が求められると申請者は考える.他者とのコミュニケーション能力,プレゼン能力を高め自身が興味深いと感じるアイデアを生き生きと伝えられるようになりたい.


\noindent
\fbox{\textbf{3.挑み続ける粘り強さ}}

長い期間全力で数学の研究を続けるためには.数学の技能に加え粘り強さが重要である.数学を好きで居続ける,情熱を燃やし続けることが大切である.チャレンジングな課題を見つけ続け,さまざまな分野の新しい知識に触れそして感動をする経験を多くするべきである.そのため数学の微分幾何学以外はもちろんその他の分野の学習も欠かさず,アンテナを張っておくことが大切と申請者は考えている.

%end 目指す研究者像 ====================

\vspace{5mm}
\noindent
\textbf{(2)上記の「目指す研究者像」に向けて、特別研究員の採用期間中に行う研究活動の位置づけ}


%begin 研究活動の位置づけ ====================

本研究活動は上の3つの能力を向上してくれると申請者は考えている.以下に理由を述べる.


\noindent\textbf{1. }80年代にNovikovにより生まれた本研究テーマはシンプルな拡張にもかかわらず広域な分野との繋がりがあり,また最先端の理論であるフレアーホモロジーの土台になっている.シンプルで応用が広いこの理論はモース理論の本質に深く迫る理論であり,数学のその他の分野に対しても見本になるような一般化の例に申請者は思えた.そのような理論に触れることが申請者が求める能力の1つ目を向上させるであろう.


\noindent\textbf{2. }問題を解決する能力を高めるため,研究集会等に参加し個人の数学の技能を磨くことに加え本研究活動の間に得られた結果を積極的に発信していきたいと考えている.学会での発表や国内外の研究者との議論をたくさんこなし数学のアイデアを共有する経験を重ねたい.研究集会を重ね数学の楽しみを共有できる友人を増やすことができれば望外の喜びである.


\noindent\textbf{3. }研究期間中にゲージ理論やシンプレクティック幾何学をはじめとした幅広いセミナーに参加することにより広い知識を獲得し刺激を得たい.また数学,物理学のほか工学にさえも本研究活動が影響を及ぼす可能性がある.近年パーシステントホモロジーとよばれるトポロジーを用いたデータ解析の手法が工学で活発に研究されている.プロットされたデータの点集合を半径を持たせた円にかえその半径を動かした時の図形のトポロジーの変化からデータの複雑さを図る手法である.この手法はモース理論,Morse Novikov理論との関係も指摘されており,それは本研究テーマの適用範囲
がとても広いことを表している.そのためMorse Novikov理論は数学,物理学やそれ以外の分野との関係が存在し,好奇心を強く満たしてくれるような魅力的な理論である.この研究が幅広い見識を与えてくれ好奇心を刺激してくれることは確実であり数多くの挑戦しがいのある課題を与えてくれると考えられる.




%end 研究活動の位置づけ ====================

\input{pieces/p05_my_ambitions_01}

%#Split: 99_tail
\input{pieces/hook9} % pieces
\end{document}
