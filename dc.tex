%\documentclass[11pt,a4paper,uplatex,dvipdfmx]{ujarticle} 		% for uplatex
\documentclass[11pt,a4j,dvipdfmx]{jarticle} 					% for platex
\input{pieces/form00_header} % pieces
\input{pieces/kakenhi7} % pieces
\input{pieces/form01_dcpd_header} % pieces
\input{pieces/hook3} % pieces
%#Name: dc
\input{pieces/form03_dcpd_headers} % pieces
\input{pieces/form04_dc_header} % pieces
% ===== Global definitions for the Kakenhi form ======================
% 基本情報
%
%------ 研究課題名  -------------------------------------------
\newcommand{\研究課題名}{象の卵}

%----- 研究機関名と研究代表者の氏名-----------------------
\newcommand{\研究機関名}{京都大学}
\newcommand{\研究代表者氏名}{福士謙二}
\newcommand{\me}{\underline{\underline{H.~Yukawa}}}
\input{pieces/inst_dcpd} % pieces
% user07_header
% ===== my favorite packages ====================================
% ここに、自分の使いたいパッケージを宣言して下さい。
\usepackage{wrapfig}
%\usepackage{amssymb}
%\usepackage{mb}
%\DeclareGraphicsRule{.tif}{png}{.png}{`convert #1 `dirname #1`/`basename #1 .tif`.png}
\usepackage{lineno}
\usepackage{amsmath}
\usepackage{amsfonts}
\usepackage{latexsym}
\usepackage{amsthm}
\usepackage{udline}
\usepackage[dvipdfmx]{graphicx}
\usepackage{here}
\usepackage{ascmac}


% ===== my personal definitions ==================================
% ここに、自分のよく使う記号などを定義して下さい。
\newcommand{\klpionn}{K_L \to \pi^0 \nu \overline{\nu}}
\newcommand{\kppipnn}{K^+ \to \pi^+ \nu \overline{\nu}}

% ----- 業績リスト用 -------------
\newcommand{\paper}[6]{%
	% paper{title}{authors}{journal}{vol}{pages}{year}
	\item ``#1'', #2, #3 {\bf #4}, #5 (#6).			% お好みに合わせて変えてください。
}

\newcommand{\etal}{\textit{et al.\ }}
\newcommand{\ca}[1]{*#1}	% corresponding author;   \ca{\yukawa}  みたいにして使う
\newcommand{\invitedtalk}{招待講演}

\newcommand{\yukawa}{H.~Yukawa}					% no underline
%\newcommand{\yukawa}{\underline{\underline{H.~Yukawa}}}	% with 2 underlines
\newcommand{\tomonaga}{S.~Tomonaga}

\newcommand{\prl}{Phys.\ Rev.\ Lett.\ }		% よく使う雑誌も定義すると楽

% ===== 欄外メモ ==================
\newcommand{\memo}[1]{\marginpar{#1}}
%\renewcommand{\memo}[1]{}	% 全てのメモを表示させないようにするには、行頭の"%"を消す

%\input{../../sample/simple/contents}	% skip
\input{pieces/hook5} % pieces

\begin{document}
\input{pieces/hook7} % pieces
%#Split: 01_background
%#PieceName: p01_background
\input{pieces/p01_background_00}
\section{研究の位置づけ}
%    <<最大 1ページ>>

%s03_background
%begin 本研究の着想に至った経緯など ====================
\noindent
\fbox{\textbf{Morse Novikov理論のこれまで}}

多様体のトポロジーをモース関数とその臨界点を用いて調べるモース理論は古くからトポロジーにおいて最も重要な理論の一つであり,のちに詳しく述べるモース理論の無限次元版のフレアーホモロジー等への一般化などその影響力は現在においても健在である.Sergei Novikovは1980年ごろ\textbf{\ul{ゼロ点の近傍でモース関数の外微分と同じ形をしているclosed 1-form(以下Morse 1-formと呼ぶ)のゼロ点を調べること}}を提唱した.これをMorse Novikov理論と呼ぶ.モース関数$f$に対し$df$のゼロ点と$f$の臨界点は一致するため,Morse Novikov理論はモース理論の一般化になっている.古典的な閉多様体上のモース理論の主な成果に,モース関数の臨界点から閉多様体と同じホモトピー型のCW複体を構成することや,臨界点たちを生成元とするチェイン複体で,勾配$-\text{grad} f$に沿った,臨界点同士を結ぶgradient flowの本数をバウンダリー写像としたホモロジー理論の構築がある.
Morse Novikov理論の始まりにより,Morse 1-formに対してもこれらと類似した結果が得られることがわかったが,\textbf{\ul{その際のモース理論との一番の違いはgradient flowの挙動である}}.
\begin{wrapfigure}[11]{r}[0mm]{10zw}
    \vspace*{-\intextsep} %
	\includegraphics[width=0.4\textwidth]{figure3.pdf}
	\caption{高さ関数とコホモロジーが0でないclosed 1-formのflow}
\end{wrapfigure}
closed 1-formに対してもflowは定義される.モース理論においてはflowに沿って関数値が単調に下がっていくが,Morse 1-formの場合flowの挙動は簡単でなく,ホモロジー理論を作ろうにも,臨界点を結ぶflowが無限に存在したり,flowが収束せず同じ軌道を周り続けることがある.この問題は,形式的無限和を許す完備化された係数環(Novikov環と呼ぶ)とflowたちを対応させて解決され,Novikov環係数のチェイン複体(Novikov複体と呼ぶ)が開発された.flowの挙動の複雑さがこの理論の一番のネックであるが,複雑さからくる興味深い結果もまた存在する.それは,flowの始点と終点が同じになるhomoclinic cycleという軌道たちが,ある条件下で存在するという力学系的に非常に興味深い結果である.このように\textbf{\ul{この理論は,これまで主にモース理論のアナロジーやgradient flowの観点から深く研究されてきた}}.



\noindent
 \fbox{\textbf{当分野における課題について}}

Morse Novikov理論のモース理論のアナロジーとしての理論の構築がひと段落したところで,Novikovが始めた当初の,\textbf{\ul{モース理論の多重値関数への一般化の理念をさらに追求していくこと}}が今後の課題であり,特異点をもつ多様体へのMorse Novikov理論の拡張がNovikovの理念の深化につながると申請者は考えている.
Novikovは,closed 1-form $\omega([\omega]\neq0)$には積分値が0でない閉曲線$\gamma$が存在することから,多価関数$f$を用いて$\omega=df$のとき(厳密な式ではない) $f(\gamma(1))-f(\gamma(0)) = \int_{\gamma}\omega\neq0$となり,$\gamma$のループで$f$の分岐がずれたと見ることができるので,$\omega$を多価関数を外微分したようなものと考えた.最も簡単な例は$S^1$とその角度$\theta$である.$\theta$は一周すると0から$2\pi$まで動く多価関数だが,角度形式と言う$d\theta$は定義される.この洞察から,\textbf{\ul{分岐や特異点のモノドロミー作用の研究にMorse Novikov理論が関わりがありそうなことがわかる}}.本研究活動では,
\textbf{\ul{特異点周りのclosed 1-formのflowの挙動など,特異点の周りの情報をもつ多重値モース理論に由来する不変量を,微分幾何学の解析的なアプローチから考察したい}}.


\noindent
	\fbox{\textbf{着想の経緯}}

	申請者は大学院で物理学科から数学科へ入学し,入学後は微分幾何学やリーマン幾何学,モース理論を特に勉強しつつMorse Novikov理論のテキストを通読した.そしてこの理論は,モース理論から生まれた幾何学の華々しい結果たちを精緻な結果で蘇らせ,さらに無限次元モース理論の土台となる意義深い理論であると確信した.大学院で学科を変えたように申請者は,物理学から数学までに至るさまざまな分野の繋がりに幅広く関心を抱いており,本研究活動を通して,\textbf{\ul{特にトポロジーと解析学との深い関係の解明という大仕事の一翼を担えると考えた}}.




%end 本研究の着想に至った経緯など ====================

\input{pieces/p01_background_01}

%#Split: 02_purpose_plan
%#PieceName: p02_purpose_plan
\input{pieces/p02_purpose_plan_00}
\section{研究目的・内容等}
%    <<最大 2ページ>>

%s02_purpose_plan_dcpd
%begin 研究目的と研究計画short留意事項なし ====================
\noindent
\fbox{\textbf{研究目的}}

幾何学的対象にホモロジーなどの不変量を対応させ,分類することは昔から幾何学の関心の中心であり,多様体の位相的な不変量を,解析的な不変量と対応させる理論もまた昔から考えられてきた.微分形式と外微分から定義されるde Rhamコホモロジーはその好例である.これが発展し,de Rhamコホモロジーと偏微分方程式の解を結びつける調和積分論が存在する.外微分$d$及び微分形式の計量に関する共役作用素$\delta:\Omega^{*+1}(M)\rightarrow \Omega^{*}(M)$から定まるラプラシアン$\Delta:=\delta d + d\delta:\Omega^{*}(M)\rightarrow\Omega^{*}(M)$に対し,そのker$(\Delta)$の次元とde Rhamコホモロジーの次元が等しくなることが主結果である.
さらに,そのラプラシアンをモース関数を用いて摂動させ,モース理論の最も重要な結果である,多様体のベッチ数とモース関数の臨界点の個数を結びつけるモース不等式を解析的に導く,Witten deformationが存在する.ラプラシアンの解をモース関数の臨界点の近くに局所化させるテクニックが本質である.このような,解析学とトポロジーの関係の流れに続くものとして,特異点を持つ空間での解析学の手法を交えつつ,\textbf{\ul{Morse Novikov理論を特異点を持つ多様体へ一般化し,多重値モース理論の不変量に対応する解析的な不変量を研究したい}}.




\noindent
\fbox{\textbf{研究内容}}

上の研究目的を踏まえ本研究活動では以下の3つの項目に取り組む.


\begin{screen}
\textbf{1. }\textbf{\ul{Morse Novikov理論の境界や特異点を持つ多様体への拡張}}

\begin{wrapfigure}[8]{r}[0mm]{12zw}
    \vspace*{-\intextsep} %
	\includegraphics[width=0.3\textwidth]{figure2.pdf}
	\caption{境界でのflowの挙動}
\end{wrapfigure}

円錐計量と呼ばれる,特異点の周りで扱いやすい形をしているリーマン計量を用い,closed 1-formを用いた摂動でWitten deformationの一般化をしたい.\textbf{\ul{具体的には,特異点が有限個の点である[3]の結果を,特異点が結び目の場合など次元を持つ状況へ拡張する}}.
多様体の境界を余次元1の特異点集合と思えば,境界付き多様体への一般化も同じ枠組みで扱うことができる.境界付き多様体のモース理論は,gradient flowが境界に対し垂直に出入りするtransverseか,境界から出て行かないように,flowが近づくにつれ接するようになるtangentであるかで大別される(右図).flowがtransverseな場合のMorse Novikov理論の拡張は[1]が存在する.[2]によると,境界付近が円錐計量ならばflowはtangentになるため,円錐計量でのMorse Novikov理論の一般化という枠組みから,flowがtangentな境界つき多様体のモース理論への応用も得られる.


\end{screen}

\noindent
\begin{screen}
\textbf{2. } \textbf{\ul{特異点を持つ多様体の位相不変量への新たなアプローチの開発}}


Witten deformationの顕著な応用例に,Cheeger-M\"{u}llerの定理の別証明がある.これは,多様体のホモロジーのチェイン複体と基本群の線形表現から定まるReidemeister torsionと,ラプラシアンの固有値たちのゼータ関数から定まるAnalytic torsionが等しくなるという結果である.位相的な量と解析的な量が一致することを示した,現在の微分幾何学においても見本のような定理である.この定理に触発された,解析的に定義された量から位相的な量を取り出す理論は指数定理やゲージ理論と呼ばれる分野で数多く存在する.\textbf{\ul{結び目補空間に対するNovikov複体に付随するReidemeister torsionが[4]で定義されており,これに対応する解析的な量をWitten deformationを用いて調べる}}.



\end{screen}

\noindent
\begin{screen}
\textbf{3. } \textbf{\ul{フレアーホモロジーと解析的な不変量の研究}}

多重値関数に対してのモース理論というコンセプトは,フレアーホモロジーへも大きな影響を与えている.フレアーホモロジーというのは数多くの種類があるが,モジュライなどの無限次元の対象に汎関数をあたえ,汎関数の臨界点たちを生成元とするチェイン複体のホモロジーを考えるというアイデアは共通している.臨界点を結ぶgradient flowの本数がバウンダリー写像である.\textbf{\ul{2のReidemeister torsionとも関わりがある,多重値モース理論に対応したヒーガード分解が存在し,これに関連するsuturedフレアーホモロジーと解析的な量の関わりを調べたい}}.
ヒーガード分解とは,円板に取手をつけたようなハンドル体とよばれるものを2つ,境界で張り合わせたもので3次元多様体を表すことである.



\end{screen}


\noindent
\fbox{\textbf{研究手法}}
\noindent

それぞれの研究内容に対し用いる手法を述べる.

\noindent
\textbf{1. }特異点を持つ多様体の調和積分論,Witten deformation

\noindent
\textbf{2. }Reidemeister torsion,Analytic torsion,Cheeger-M\"{u}llerの定理,Witten deformation

\noindent
\textbf{3. }ヒーガードフレアーホモロジー理論,微分幾何学の解析の発展的な手法


\noindent
\fbox{\textbf{研究計画}}

上の3つの研究内容は順番につながっている.\textbf{1}で,Morse Novikov理論とWitten deformationを特異点を持つ空間へ一般化し,\textbf{2}では,\textbf{1}の結果を用いて結び目補空間におけるNovikov複体の不変量と対応する解析的な量を研究する.結び目の周りのgradient flowの挙動などが関わる,新たな不変量が期待される.\textbf{3}では,多重値モース理論から定まる不変量の解析的なバージョンを調べるという考えを推し進め,多重値モース理論から定まる,フレアーホモロジーに対応する解析的な量について研究する.

\noindent
\textbf{博士後期課程1年目. }[1],[2]の論文はセミナーで読み終えたため,実際にこの論文たちのテクニックを用いて,\textbf{\ul{結び目などの特異点を持つ多様体へのWitten deformationの拡張}}を試みたい.この方向性の研究は修士課程のこれからと博士後期課程初年度で行うつもりである.これに加え,flowがtangentな境界付き多様体のNovikov複体を構成することを博士後期課程初年度では行う.

\noindent
\textbf{博士後期課程2年目. }
\textbf{\ul{Morse Novikov理論に関連する位相不変量に解析的な量を対応させたい}}.研究内容\textbf{2}に沿って,
Cheeger-M\"{u}llerの定理の特異点を持つ多様体への拡張を行うとともに,Novikov複体におけるReidemeister torsionとAnalytic torsionの関係を考えたい.

\noindent
\textbf{博士後期課程3年目. }2年目に引き続き,\textbf{\ul{多重値モース理論に関連するフレアーホモロジーを解析的な量と対応させる}}.Witten deformationや,さらに発展した手法であるゲージ理論と呼ばれる分野で定義される解析的な量が関わることを想定している.この結果を博士後期課程の最終的な目標とする.


\noindent
\fbox{\textbf{独創性}}

特異点を持つ多様体へのMorse Novikov理論の一般化は現状進んでおらず,
flowがtangentな境界付き多様体のモース理論へのMorse Novikov理論の一般化を言及している論文は存在しない.この多重値モース理論の不変量の解析学との繋がりを調べる新しい研究は,特異点を持つ多様体の新たな不変量の提供や,トポロジーと,微分幾何学や力学系を巻き込んだ壮大な結果が得られると期待される.



\vspace{2mm}
\noindent \textbf{参考文献}

\noindent
[1] LI TIEQIANG,TAN. Topology of Closed 1-Forms on Manifolds with Boundary. Durham University Doctoral thesis, 2009.


\noindent
[2] Akaho Manabu. MORSE HOMOLOGY AND MANIFOLDS WITH BOUNDARY. World Scientific, 2007.

\noindent
[3] Ursula Ludwig. An Extension of a Theorem by Cheeger and M\"{u}ller to Spaces with Isolated Conical Singularities. Comptes Rendus Mathematique, 2018.

\noindent
[4] Hiroshi GODA and Andrei V. PAJITNOV. Dynamics of gradient flows in the half-transversal Morse theory. Proceedings of the Japan Academy Series A Mathematical Sciences, 2009.


%end 研究目的と研究計画short留意事項なし ====================
\input{pieces/p02_purpose_plan_01}

%#Split: 03_rights
%#PieceName: p03_rights
\input{pieces/p03_rights_00}
\section{人権の保護及び法令等の遵守への対応}
%    <<最大 1ページ>>

% s09_rights
%begin 人権の保護及び法令等の遵守への対応 ====================
手続きが必要な研究は行わない予定である.

%end 人権の保護及び法令等の遵守への対応 ====================

\input{pieces/p03_rights_01}

%#Split: 04_abilities
%#PieceName: p04_abilities
\input{pieces/p04_abilities_00}
\section{研究遂行力の自己分析}
%    <<最大 2ページ>>

% s14_abilities
%begin 自己分析 ====================
%\DCPDInstructionsA\\% <-- 留意事項:これは消すか、コメントアウトしてください。

\noindent
\textbf{(1) 研究に関する自身の強み}
%\DCPDInstructionsB% <-- 留意事項:これは消すか、コメントアウトしてください。

申請者は研究を遂行するにあたってのさまざまな強みを有している.一番の強みは幅広い知識量や積極性であり,それは申請者が学部では物理学を,そして修士では数学を専攻し幅広い知識を有していることから言える.
その知識量から複眼的な考察が可能であり,数学や物理学のさまざまな分野の架け橋になる研究者としての資質を有している.以下に自身の強みに対して詳細を述べたい.



\noindent
\fbox{\textbf{主体性}}

申請者は京都大学の物理学科を卒業したのち修士課程で同大学の数学科へ入学した.それは物理学の定式化に用いられる洗練された数学の理論に魅力を覚えたからである.物理学の講義を学部のうちに一通り履修し,大学の数学も勉強はしていたものの数学科の講義全てに出席していたわけでなかった.そのため図書館で必死に大学院入試の勉強をこなし合格し,その後も筆記試験と面接を経て大学内の博士後期課程へ進学するコースに認められることができた.これは申請者が幅広い関心をもち自ら動く能力を有していることを表している.そのほか大学内の計算機の管理のオフィスアシスタントの活動,ティーチングアシスタントという学部生の微分積分学や線形代数学の課題の採点の仕事を積極的にこなした.これらの経験から申請者は主体的に動く力を十分に持っていると考えられる.


\noindent
\fbox{\textbf{知識量}}




申請者は学部時代より数学や物理学の学習を積極的に行ってきた.物理学科のころは電磁気学,統計力学,量子力学に加え一般相対性理論をよく学んだ.研究室は宇宙物理学の研究室で前期は一般相対性理論のテキストの輪読,後期は惑星と衛星が潮汐力により自転の周期が同期する現象の数値計算の卒業研究を行った.大学院からは数学科へ移り普段のセミナーをこなしつつ微分幾何学,調和積分論,特性類,ゲージ理論の知識をキャッチアップした.物理学科のころから現在に至るまでの学習の過程はつながっている.まず物理学の学習を通して,物理学を記述する言語としての多様体論,微分幾何学,ゲージ理論に関心を抱いた.現代の物理学は一般相対性理論と量子力学が2つの柱であり,それらの理論の定式化にはそれぞれ深い数学が用いられる.相対性理論は光速度不変という原則を空間と時間を入れた計量の入った4次元の多様体の幾何学で定式化することで確立された.
また現代の力学はラグランジアンとよばれる汎関数を変分して得られる方程式が物理現象を表すという最小作用の原理に基づいた理論構築がなされており,量子力学もそのアナロジーで定式化が進んでいる.そしてゲージ群と呼ばれる対称性を課したラグランジアンを扱うという物理学におけるゲージ理論は,ゲージ群$G$をファイバーにもつ主$G$束を扱う理論として数学的に定式化された.申請者はこのような物理学を数学的に明晰に扱う様を魅力に感じた.数学としてのゲージ理論の花形であるドナルドソン理論とは主$G$束の共変微分を表す接続と呼ばれる対象全体の空間に不変量を見出すという,ベクトル束の不変量を扱う特性類の考えを発展させた理論である.インスタントンフレアーホモロジーはさらに調和積分などの解析の手法を用いて接続全体の空間に汎関数を定義し無限次元モース理論を遂行させて定義される.フレアーホモロジー理論の構築にはモース理論のさまざまな技術が応用されておりMorse Novikov理論も例外ではない.このように申請者の過去の学習は現在の数学の最先端の理論までつながっている.
さらにNovikovの多重値のモース理論というアイデアも数理物理,ハミルトン力学系のフローのゼロ点の数の評価などから生まれたアイデアであり,Witten deformationも場の理論における超対称性とよばれる理論とモース理論との関係を追求する中で生まれたものである.このように申請者の本研究活動は数学と物理学に範囲が広く及ぶ遠大なテーマである.そのため数学科の学生に比べ幅広く学習を行い,新たな視点を提供できるであろう申請者は本研究活動において格好の人物である.\textbf{\ul{特にNovikovの元々の数理物理へのアプローチは現在のMorse Novikov理論の主流ではないように感じるため,本研究活動のほかにオリジナリティのある課題を見つけることができる可能性が高いだろう}}.


\noindent
\fbox{\textbf{セミナーの主催者としての能力}}

申請者は京都大学内でオフィスアシスタントと呼ばれる計算機の管理の仕事を修士から行っている.プログラミングにもとより関心が強く学習もしておりそのような仕事に関心があった.主な仕事内容は数学科のwebページの管理をはじめ,PCやプリンタ等の大学内で使用する機器の使用方法のマニュアルの作成などである.大学内のサーバーの構造や通信の仕組みについて日々の業務によりその知識を深めている.新型コロナウイルスはセミナーや研究集会の在り方を大幅に変化させ,オンライン化の指向性を極めて高めた.今後もこの流れは変わらないと考えられるため,オフィスアシスタントで得た知識や経験は貴重なものであり,オンラインで開催されたセミナーでのトラブルの対応など研究集会の主催者として適切な能力を有していると考えられる.



\noindent
\fbox{\textbf{向上心}}


申請者は大学院で物理学から数学へ専攻を変えたように,極めて強い好奇心や向上心,粘り強さで数学,物理学,プログラミングの学習や研究活動に取り組んできた.この飽くなき探究心でさらなる分野の発展に貢献したい.学部生の頃の実験の演習では,長らく器具のずれにより実験結果がずれる事象があったが,それを補正するモデルを創出することを数学力を持っていることから班員から任せられたように,広い知識を持つがゆえに頼れることを経験してきた.数学の分野においてもこの問題の解決力,創造力を活かしたい.

学習面においては,\textbf{\ul{大学院の講義で優秀な成績を収め,授業料は全て免除され,京都大学内の理学研究科数学・数理解析専攻学生に対する支援に採用され,支援金が給付された}}.これは申請者が大学院で優秀な成績を修めてきたことを表している.
普段の講義や研究室内のセミナーに加え,大学内の微分トポロジーセミナーや関西ゲージ理論セミナー,外部の研究集会では微分トポロジー'21,名古屋大学で主催された「4次元多様体I,II」の輪読会等,多くの研究集会へ積極的に参加してきた.これらの知識は今後の研究者としての活動に大切なものであり,これからも積極的に研究集会へ参加をしていきたい.





\vspace{5mm}
\noindent
\textbf{(2) 今後研究者として更なる発展のため必要と考えている要素}


研究者としての更なる発展には,数学のより深い知識,分野を横断するような幅広い見識が必要である.そしてそれを可能にするのは互いに教え合う友人,自身の考えをわかりやすく伝える能力であると申請者は考える.


\noindent
\fbox{\textbf{知識や技能の拡充}}

論文やテキストを熟読することを継続していきたい.現在は自分で関心のある論文に目を通してそれを指導教官とのセミナーで発表をしているが,今後は論文の輪読会を企画していくことで知識や知人を増やせると考えている.



\noindent
\fbox{\textbf{幅広い交友関係}}


異なる分野の知識が必要になるときなどに,互いに教え合ったり刺激しあう学友が必要なことがある.ゲージ理論では微分幾何学で定式化された公式を,実際に興味深い例に当てはめるときに代数幾何学などの知識が必要になることがある.また複素幾何学とシンプレクティック幾何学の双対性のミラー対称性や調和写像の分野では微分幾何学と代数幾何学の繋がりは昔から強く意識されている.そのような状況に対応するには幅広い知識,友人が必要である.特に申請者は大学院から数学を専攻しているため知り合いが多いわけでないため,不慣れな分野で迷い込まないよう的確なアドバイスをもらえる教えてくれる先達,同じ分野の志を共にする友人を多く持ちたい.セミナーや研究集会に積極的に参加し発言もしっかりしていくことで交友関係は広がっていく.


\noindent
\fbox{\textbf{伝達力の更なる向上}}


セミナーでのプレゼンテーションの能力の向上を目指したい.これは自らが学会やセミナーで発表することを重ねることで身についていくものである.セミナーやその後の議論で自分のアイデアを円滑に伝えられることは研究活動において非常に重要である.発表中の伝わりやすい言葉や話し方,スライドの作り方にこだわりながらセミナー発表をこなしていきたい.


%end 自己分析 ====================

\input{pieces/p04_abilities_01}

%#Split: 05_my_ambitions
%#PieceName: p05_my_ambitions
\input{pieces/p05_my_ambitions_00}
\section{目指す研究者像等}
%    <<最大 1ページ>>

% s17_my_ambitions
\noindent
\textbf{(1)目指す研究者像 {\footnotesize ※目指す研究者像に向けて身に付けるべき資質も含め記入してください。}}

%begin 目指す研究者像 ====================
申請者の理想の研究者像は「\textbf{幾何学およびその周辺分野の発展に貢献し続ける研究者}」である.自分が魅入られた数学,特に微分幾何学で最先端の研究を行いたい.それだけでなく常に全力で好奇心を保ち研究生活を駆け抜けたい.申請者はそのような研究者になるために以下の能力を兼ね備えた研究者になりたいと考える.


\noindent
\fbox{\textbf{1.深掘りする思考力}}

数学の最大の魅力は理論が抽象化されているが故の適応範囲の広さであると申請者は考える.数学が適応範囲が広い学問であるということは古くから科学の発展に数学が寄与してきたことが示している.抽象化されたインパクトの大きい理論を構築するためには現象の本質を抽出する能力が必要であり,透徹した理論に多く触れることで本質を見通す感覚を養われると申請者は考えている.


\noindent
\fbox{\textbf{2.問題を解決する能力}}

本質を抽出する能力が問題の解決の糸口をみつけたりあらたな理論を構築するアイデアを見つけてくれるならば,そのアイデアを遂行する能力も必要である.自身が数学に造詣が深いことや計算力を身に着けることももちろん重要だが,とりわけ他人と協力し問題を解決する能力が求められると申請者は考える.他者とのコミュニケーション能力,プレゼン能力を高め自身が興味深いと感じるアイデアを生き生きと伝えられるようになりたい.


\noindent
\fbox{\textbf{3.挑み続ける粘り強さ}}

長い期間全力で数学の研究を続けるためには.数学の技能に加え粘り強さが重要である.数学を好きで居続ける,情熱を燃やし続けることが大切である.チャレンジングな課題を見つけ続け,さまざまな分野の新しい知識に触れそして感動をする経験を多くするべきである.そのため数学の微分幾何学以外はもちろんその他の分野の学習も欠かさず,アンテナを張っておくことが大切と申請者は考えている.

%end 目指す研究者像 ====================

\vspace{5mm}
\noindent
\textbf{(2)上記の「目指す研究者像」に向けて、特別研究員の採用期間中に行う研究活動の位置づけ}


%begin 研究活動の位置づけ ====================

本研究活動は上の3つの能力を向上してくれると申請者は考えている.以下に理由を述べる.


\noindent\textbf{1. }80年代にNovikovにより生まれた本研究テーマはシンプルな拡張にもかかわらず広域な分野との繋がりがあり,また最先端の理論であるフレアーホモロジーの土台になっている.シンプルで応用が広いこの理論はモース理論の本質に深く迫る理論であり,数学のその他の分野に対しても見本になるような一般化の例に申請者は思えた.そのような理論に触れることが申請者が求める能力の1つ目を向上させるであろう.


\noindent\textbf{2. }問題を解決する能力を高めるため,研究集会等に参加し個人の数学の技能を磨くことに加え本研究活動の間に得られた結果を積極的に発信していきたいと考えている.学会での発表や国内外の研究者との議論をたくさんこなし数学のアイデアを共有する経験を重ねたい.研究集会を重ね数学の楽しみを共有できる友人を増やすことができれば望外の喜びである.


\noindent\textbf{3. }研究期間中に,微分幾何学のゲージ理論やシンプレクティック幾何学をはじめとした幅広いセミナーに参加することにより広い知識を獲得し刺激を得たい.また数学,物理学のほか工学にさえも本研究活動が影響を及ぼす可能性がある.近年,パーシステントホモロジーとよばれるトポロジーを用いたデータ解析の手法が工学で活発に研究されている.プロットされたデータの点集合を半径を持たせた円にかえ,その半径を動かした時の図形のトポロジーの変化からデータの複雑さを図る手法である.この手法はモース理論,Morse Novikov理論との関係も指摘されており,それは本研究テーマの適用範囲
がとても広いことを表している.そのためMorse Novikov理論は数学,物理学やそれ以外の分野との関係が存在し,好奇心を強く満たしてくれるような魅力的な理論である.この研究が幅広い見識を与えてくれ,好奇心を刺激してくれることは確実である.




%end 研究活動の位置づけ ====================

\input{pieces/p05_my_ambitions_01}

%#Split: 99_tail
\input{pieces/hook9} % pieces
\end{document}
