%\documentclass[11pt,a4paper,uplatex,dvipdfmx]{ujarticle} 		% for uplatex
\documentclass[11pt,a4j,dvipdfmx]{jarticle} 					% for platex
\input{pieces/form00_header} % pieces
\input{pieces/kakenhi7} % pieces
\input{pieces/form01_dcpd_header} % pieces
\input{pieces/hook3} % pieces
%#Name: dc
\input{pieces/form03_dcpd_headers} % pieces
\input{pieces/form04_dc_header} % pieces
% ===== Global definitions for the Kakenhi form ======================
% 基本情報
%
%------ 研究課題名  -------------------------------------------
\newcommand{\研究課題名}{象の卵}

%----- 研究機関名と研究代表者の氏名-----------------------
\newcommand{\研究機関名}{京都大学}
\newcommand{\研究代表者氏名}{福士 謙二   }
\newcommand{\me}{\underline{\underline{H.~Yukawa}}}
\input{pieces/inst_dcpd} % pieces
% user07_header
% ===== my favorite packages ====================================
% ここに、自分の使いたいパッケージを宣言して下さい。
\usepackage{wrapfig}
%\usepackage{amssymb}
%\usepackage{mb}
%\DeclareGraphicsRule{.tif}{png}{.png}{`convert #1 `dirname #1`/`basename #1 .tif`.png}
\usepackage{lineno}
\usepackage{amsmath}
\usepackage{amsfonts}
\usepackage{latexsym}
\usepackage{amsthm}
\usepackage{udline}
\usepackage[dvipdfmx]{graphicx}
\usepackage{here}
\usepackage{ascmac}


% ===== my personal definitions ==================================
% ここに、自分のよく使う記号などを定義して下さい。
\newcommand{\klpionn}{K_L \to \pi^0 \nu \overline{\nu}}
\newcommand{\kppipnn}{K^+ \to \pi^+ \nu \overline{\nu}}

% ----- 業績リスト用 -------------
\newcommand{\paper}[6]{%
	% paper{title}{authors}{journal}{vol}{pages}{year}
	\item ``#1'', #2, #3 {\bf #4}, #5 (#6).			% お好みに合わせて変えてください。
}

\newcommand{\etal}{\textit{et al.\ }}
\newcommand{\ca}[1]{*#1}	% corresponding author;   \ca{\yukawa}  みたいにして使う
\newcommand{\invitedtalk}{招待講演}

\newcommand{\yukawa}{H.~Yukawa}					% no underline
%\newcommand{\yukawa}{\underline{\underline{H.~Yukawa}}}	% with 2 underlines
\newcommand{\tomonaga}{S.~Tomonaga}

\newcommand{\prl}{Phys.\ Rev.\ Lett.\ }		% よく使う雑誌も定義すると楽

% ===== 欄外メモ ==================
\newcommand{\memo}[1]{\marginpar{#1}}
%\renewcommand{\memo}[1]{}	% 全てのメモを表示させないようにするには、行頭の"%"を消す

%\input{../../sample/simple/contents}	% skip
\input{pieces/hook5} % pieces

\begin{document}
\input{pieces/hook7} % pieces
%#Split: 01_background
%#PieceName: p01_background
\input{pieces/p01_background_00}
\section{研究の位置づけ}
%    <<最大 1ページ>>

%s03_background
%begin 本研究の着想に至った経緯など ====================
\noindent
\fbox{\textbf{Morse Novikov理論のこれまで}}

多様体のトポロジーをモース関数とその臨界点を用いて調べるモース理論は,古くからトポロジーにおいて最も重要な理論の一つであり,のちに詳しく述べるモース理論の無限次元版のフレアーホモロジー等への一般化などその影響力は現在においても健在である.Sergei Novikovは1980年ごろ,\textbf{\ul{ゼロ点の近傍でモース関数の外微分と同じ形をしているclosed 1-form(以下Morse 1-formと呼ぶ)のゼロ点を調べること}}を提唱した.これをMorse Novikov理論と呼ぶ.モース関数$f$に対し$df$のゼロ点と$f$の臨界点は一致するため,Morse Novikov理論はモース理論の一般化になっている.古典的な閉多様体上のモース理論の主な成果に,モース関数の臨界点から閉多様体と同じホモトピー型のCW複体を構成することや,臨界点たちを生成元とするチェイン複体で,勾配$-\text{grad} f$に沿った,臨界点同士を結ぶgradient flowの本数をバウンダリー写像としたモースホモロジーと呼ばれるホモロジー理論の構築がある.
Morse Novikov理論の始まりにより,Morse 1-formに対してもこれらと類似した結果が得られることがわかったが,\textbf{\ul{その際のモース理論との一番の違いはgradient flowの挙動である}}.


\begin{wrapfigure}[11]{r}[0mm]{10zw}
    \vspace*{-\intextsep} %
	\includegraphics[width=0.4\textwidth]{figure3.pdf}
	\caption{高さ関数とコホモロジーが0でないclosed 1-formのflow}
\end{wrapfigure}
closed 1-formに対してもflowは定義される.モース理論においてはflowに沿って関数値が単調に下がっていくが,Morse 1-formの場合flowの挙動は簡単でなく,ホモロジー理論を作ろうにも,臨界点を結ぶflowが無限に存在したり,flowが収束せず同じ軌道を周り続けることがある.この問題は,形式的無限和を許す完備化された係数環(Novikov環と呼ぶ)とflowたちを対応させて解決され,Novikov環係数のチェイン複体(Novikov複体と呼ぶ)が開発された.flowの挙動の複雑さがこの理論の一番のネックであるが,複雑さからくる興味深い結果もまた存在する.それは,flowの始点と終点が同じになるhomoclinic cycleという軌道たちが,ある位相的条件下で存在するという力学系的に非常に興味深い結果である.このように,\textbf{\ul{この理論はこれまで主にモース理論のアナロジーやflowの力学系の観点から深く研究されてきた}}.



\noindent
 \fbox{\textbf{当分野における課題について}}

Morse Novikov理論のモース理論のアナロジーとしての理論の構築がひと段落したところで,今後はMorse Novikov理論をより一般の空間へ拡張することや,Morse Novikov理論から得られる不変量をさまざまな分野と結びつけることが課題である.例えば,Novikov複体のホモロジーはflowの挙動の情報をもつため,より力学系や解析の観点からみて興味深い状況へMorse Novikov理論の拡張することはトポロジーと解析学の関係のさらなる解明につながる.
本研究活動では,
\textbf{\ul{特異点を持つ多様体に対しMorse Novikov理論を拡張し,特異点周りのflowの挙動など,特異点の周りの情報をもつMorse Novikov理論に由来する不変量を解析的に研究する}}.
特に先行研究が進んでいる,3次元多様体から結び目を除いた,結び目を特異点にもつ結び目補空間に対し,Morse Novikov理論に関連する解析的な新たな不変量を導入する.


\noindent
	\fbox{\textbf{着想の経緯}}

特異点を持つ多様体へMorse Novikov理論を一般化するという着想の経緯は,Novikovが本理論を創始した理由の\textbf{\ul{モース理論の多重値関数への一般化}}にある.
Novikovは,closed 1-form $\omega([\omega]\neq0)$には積分値が0でない閉曲線$\gamma$が存在することから,多価関数$f$を用いて$\omega=df$と表せるなら(厳密な式ではない) $f(\gamma(1))-f(\gamma(0)) = \int_{\gamma}\omega\neq0$となり,$\gamma$のループで$f$の分岐がずれたと見ることができるので,$\omega$を多価関数を外微分したようなものと考えた.最も簡単な例は$S^1$とその角度$\theta$である.$\theta$は一周すると0から$2\pi$まで動く多価関数だが,角度形式と言う$d\theta$は定義される.この洞察から,\textbf{\ul{分岐や特異点のモノドロミー作用の研究にMorse Novikov理論が関わりがありそうなことがわかる}}.





%end 本研究の着想に至った経緯など ====================

\input{pieces/p01_background_01}

%#Split: 02_purpose_plan
%#PieceName: p02_purpose_plan
\input{pieces/p02_purpose_plan_00}
\section{研究目的・内容等}
%    <<最大 2ページ>>

%s02_purpose_plan_dcpd
%begin 研究目的と研究計画short留意事項なし ====================
\noindent
\fbox{\textbf{研究目的}}

幾何学的対象を分類する強力な不変量を作ることは昔から幾何学の関心の中心であり,既存の位相不変量に対応する解析学的な量をつくる理論もまた昔からよく考えられている.特に重要な結果に,de Rhamコホモロジーと,ラプラス方程式と呼ばれる偏微分方程式の解空間という解析的な量の間の同型を導く調和積分論がある.
さらに,そのラプラス方程式をモース関数を用いて摂動させ,閉多様体のベッチ数とモース関数の臨界点の個数を結びつけるモース不等式というものを解析学的に示す,Witten deformationという手法が存在する.本研究活動の目的は,このような解析学とトポロジーの関係の流れに続き,\textbf{\ul{特異点を持つ多様体へ一般化されたMorse Novikov理論から得られる,特異点周りの情報を反映した多重値モース理論の不変量に対応する,新たな解析的な不変量を生み出すことである}}.




\noindent
\fbox{\textbf{研究内容・手法・特色}}

上の研究目的を踏まえ本研究活動では以下の3つの項目に取り組む.それぞれ,先行研究を踏まえたこれから行う研究内容,用いる研究手法,期待できる成果や問題を考える意義等本研究の特色について述べる.


\begin{screen}
\textbf{1. }\textbf{\ul{特異点を持つ多様体へのMorse Novikov理論とWitten deformationの拡張}}

\begin{wrapfigure}[8]{r}[0mm]{12zw}
    \vspace*{-\intextsep} %
	\includegraphics[width=0.3\textwidth]{figure2.pdf}
	\caption{境界でのflowの挙動}
\end{wrapfigure}

\fbox{\textbf{内容}}
 [1]で境界付き多様体へのMorse Novikov理論の拡張,
 [3]で特異点が有限個の点である場合へWitten deformationが拡張されている.これらの手法をさらに特異点が結び目の場合など次元を持つ状況へ拡張する.1-formを用いたラプラス方程式の摂動を行い,円錐計量という特異点の近くで扱いやすいリーマン計量での調和積分論を改良する.


\fbox{\textbf{手法}} 特異点を持つ多様体の調和積分論,Witten deformation

\fbox{\textbf{特色}}
 境界付き多様体のモース理論は,[1]で採用されるflowが境界に対し垂直に出入りするtransverseか,flowが近づくにつれ接するようになるtangentであるかで大別される(右図).[2]によると境界付近が円錐計量ならばflowはtangentになる.そのため多様体の境界を余次元1の特異点集合と見なせば,円錐計量でのMorse Novikov理論の一般化という枠組みから,flowがtangentな境界つき多様体のモース理論への応用も得られる.

\end{screen}

\noindent
\begin{screen}
\textbf{2. } \textbf{\ul{結び目不変量への新たなアプローチの開発}}

\fbox{\textbf{内容}}
Witten deformationの応用例にCheeger-M\"{u}llerの定理の別証明がある.これは多様体のホモロジーのチェイン複体と基本群の線形表現から定まるReidemeister torsionと,ラプラシアンの固有値たちのゼータ関数から定まるAnalytic torsionが等しくなるという定理である.\textbf{1}で拡張されたWitten deformationを用いてCheeger-M\"{u}llerの定理を結び目などの特異点をもつ多様体へ拡張する.そして\textbf{\ul{[4]で定義された結び目補空間に対するNovikov複体に付随するReidemeister torsionに対応するAnalytic torsionを研究する}}.

\fbox{\textbf{手法}}Witten deformation,Analytic torsion,Reidemeister torsion,Cheeger-M\"{u}llerの定理

\fbox{\textbf{特色}}
3次元多様体の固定点が結び目になる有限群作用の商など,結び目を特異点にもつ多様体は重要な例が多くある.本研究により結び目周りのモノドロミーを反映した精緻な結び目不変量が得られる.




\end{screen}

\noindent
\begin{screen}
\textbf{3. } \textbf{\ul{2で構築した不変量に対応するフレアーホモロジー理論の構築}}

\fbox{\textbf{内容}}
フレアーホモロジーというのは数多くの種類があるが,モジュライなどの無限次元の対象に汎関数をあたえ,汎関数の臨界点たちを生成元とするチェイン複体で臨界点を結ぶgradient flowの本数をバウンダリー写像とするホモロジー理論,という構成は共通している.
またヒーガード分解とは,円板に取手をつけたようなハンドル体とよばれるものを2つ,境界で張り合わせたもので3次元多様体を表すことであり,モース理論を用いて分解が得られる.このヒーガード分解に付随するヒーガードフレアー理論というものが存在する.そして,多重値モース理論から得られる一般化されたヒーガード分解に付随するsuturedフレアーホモロジーは,オイラー数を取ると\textbf{2}のReidemeister torsionを復元する.\textbf{\ul{このsuturedフレアー理論の対応物となる,2で構成したAnalytic torsionを復元する新たなフレアー理論を構築する}}.


\fbox{\textbf{手法}}ヒーガードフレアー理論,微分幾何学の解析の発展的な手法

\fbox{\textbf{特色}}新たなフレアー理論の創出により,より高級な不変量が得られたり,種々のフレアーホモロジー理論の間の関連性の理解が進む.



\end{screen}


\noindent
\fbox{\textbf{研究計画}}

年次ごとの具体的な研究活動内容について述べる.
上の3つの研究内容は順番につながっているため,採用年次と対応して上記の研究内容に順番に着手していく.\textbf{1}で,Morse Novikov理論とWitten deformationを特異点を持つ空間へ一般化し,\textbf{2}で\textbf{1}の結果を用いて結び目補空間におけるNovikov複体の不変量と対応する解析的な量を研究する.結び目の周りのflowの挙動などが関わる,新たな不変量が期待される.\textbf{3}では,\textbf{2}で得られたAnalytic torsionを含有するより強い解析的な不変量を構築する.

\noindent
\textbf{\ul{採用1年目. }}[1],[2],[3]の論文はセミナーで既に読み終えたため,この論文たちのテクニックを用いて研究内容\textbf{1}の内容である,結び目などの特異点を持つ多様体へのMorse Novikov理論およびWitten deformationの拡張に博士前期課程の今から採用一年目の間まですぐにとりかかる.

\noindent
\textbf{\ul{採用2年目. }}
研究内容\textbf{2}に沿って,
[3]の著者であるUrsula Ludwigとコンタクトをとり,Cheeger-M\"{u}llerの定理の特異点を持つ多様体への一般化を行う.その後,Novikov複体におけるReidemeister torsionと対応するAnalytic torsionの関係を研究する.


\noindent
\textbf{\ul{採用3年目. }}\textbf{\ul{研究内容3に沿って2年目に構築したAnalytic torsionを復元するフレアー理論を構築する}}.suturedフレアーホモロジーの原論文[5]を解読するセミナーをヒーガードフレアーホモロジーの研究者と開催する.
\textbf{3}で構築するフレアー理論はゲージ理論と呼ばれる微分幾何学の理論を用いて定式化されるフレアー理論になることを想定しており,suturedフレアーホモロジーを理解したのちはゲージ理論の研究者と共同研究を行う.この結果を本研究活動の最終的な目標とする.


\noindent
\fbox{\textbf{独創性}}

特異点を持つ多様体へのMorse Novikov理論の一般化は現状進んでいない.
この多重値モース理論の不変量の解析学との繋がりを調べる新しい研究は,特異点を持つ多様体の新たな不変量の提供や,トポロジーと,微分幾何学や力学系を巻き込んだ壮大な結果をもたらすと期待される.



\vspace{2mm}
\noindent \textbf{参考文献}

\noindent
[1] LI TIEQIANG,TAN. Topology of Closed 1-Forms on Manifolds with Boundary. Durham University Doctoral thesis, 2009.


\noindent
[2] Akaho Manabu. MORSE HOMOLOGY AND MANIFOLDS WITH BOUNDARY. World Scientific, 2007.

\noindent
[3] Ursula Ludwig. An Extension of a Theorem by Cheeger and M\"{u}ller to Spaces with Isolated Conical Singularities. Comptes Rendus Mathematique, 2018.

\noindent
[4] Hiroshi GODA and Andrei V. PAJITNOV. Dynamics of gradient flows in the half-transversal Morse theory. Proceedings of the Japan Academy Series A Mathematical Sciences, 2009.

\noindent
[5] Juhász, András. Floer homology and surface decompositions. Geom. Topol. 12 no. 1, 299–350, 2008.


%end 研究目的と研究計画short留意事項なし ====================
\input{pieces/p02_purpose_plan_01}

%#Split: 03_rights
%#PieceName: p03_rights
\input{pieces/p03_rights_00}
\section{人権の保護及び法令等の遵守への対応}
%    <<最大 1ページ>>

% s09_rights
%begin 人権の保護及び法令等の遵守への対応 ====================
\noindent
本研究は該当しない

%end 人権の保護及び法令等の遵守への対応 ====================

\input{pieces/p03_rights_01}

%#Split: 04_abilities
%#PieceName: p04_abilities
\input{pieces/p04_abilities_00}
\section{研究遂行力の自己分析}
%    <<最大 2ページ>>

% s14_abilities
%begin 自己分析 ====================
%\DCPDInstructionsA\\% <-- 留意事項:これは消すか、コメントアウトしてください。

\noindent
\textbf{(1) 研究に関する自身の強み}
%\DCPDInstructionsB% <-- 留意事項:これは消すか、コメントアウトしてください。

申請者は本研究を遂行するにあたってのさまざまな強みを有している.一番の強みは,\textbf{\ul{申請者が学部では物理学を,そして修士では数学を専攻していることによる主体性と幅広い知識}}である.
申請者は数学,物理学などで生じる分野間の繋がりに強い関心があり,これが本研究活動でトポロジーと解析学のつながりを研究する理由である.
またその中でもモース理論に関心を抱いた理由は,まず物理学の学習を通して物理学の定式化に用いられる微分幾何学にまず関心を抱き,その微分幾何学をよく用いる,トポロジーに長年深い影響を与えてきたモース理論に興味を持ったからである.モース理論は古くからある理論で,それゆえ解析学や物理学の様々な分野とかかわりがあるため,申請者の関心と合致し,長所を存分に活かせる理論である.
そのため数学科の学生に比べ幅広く学習を行い,新たな視点を提供できるであろう申請者は本研究活動を行う格好の人物である.
さらなる申請者の研究に関する強みについて,以下で具体的に述べる.


\noindent
\fbox{\textbf{主体性}}

申請者は京都大学の物理学科を卒業したのち,修士課程で同大学の数学科へ入学した.それは物理学の定式化に用いられる,洗練された数学の理論に魅力を覚えたからである.物理学の講義は学部のうちに一通り履修したが,学部の数学は勉強はしていたものの知識には穴がある状態であった.そのため必死に大学院入試の勉強をこなし合格し,その後も,\textbf{\ul{筆記試験と面接を経て大学内の博士後期課程へ進学するコースに認められることができた}}.これは申請者が幅広い関心をもち,自ら動く能力を有していることを表している.
また普段の講義や研究室内のセミナーに加え,大学内の微分トポロジーセミナーや関西ゲージ理論セミナー,外部の研究集会では微分トポロジー'21,名古屋大学で主催された「4次元多様体I,II」の輪読会等,多くの研究集会へ積極的に参加してきた.
そのほか\textbf{\ul{大学内の計算機の管理のオフィスアシスタントの活動,ティーチングアシスタントという学部生の微分積分学や線形代数学の課題の採点の仕事}}を積極的にこなした.
これらの経験から申請者は主体的に動く力を十分に持っていると考えられる.


\noindent
\fbox{\textbf{知識量}}


申請者はこれまで数学や物理学の学習を積極的に行ってきた.物理学科のころはカリキュラムに沿って電磁気学,統計力学,量子力学に加え一般相対性理論をよく学び,大学院からは数学科へ移り,普段のMorse Novikov理論のテキストのセミナーをこなしつつ微分幾何学,調和積分論,特性類,ゲージ理論について学んだ.前述の通り物理学科のころから現在に至るまでの学習の過程はつながっている.

\begin{wrapfigure}[10]{r}[0mm]{19zw}
    \vspace*{-\intextsep} %
	\includegraphics[width=0.4\textwidth]{figure9.pdf}
	\caption{分野の関連図}
\end{wrapfigure}

また右図の幾何学の諸分野の結びつきは,物理学の考えを援用して得られたものであり,本研究テーマもこの結びつきに関わっている.それぞれ解説する.

ベクトル束は多様体に付属する情報を束ねた空間であり,例えば接ベクトルを全て束ねた多様体の接束は多様体の微分構造の情報を持っている.このため多様体のある構造の分類というのはベクトル束の分類に帰着され,そこからベクトル束の不変量を調べる特性類が生まれた.
ゲージ理論の花形のドナルドソン理論やフレアーホモロジーは,物理学から着想を得たベクトル束の共変微分の仕方を全て束ねたモジュライ空間を調べるという,特性類の考え方を解析学を用いて実行し不変量を得る理論である.
それぞれ非線形な調和積分論,モース理論の手法が用いられる.
Witten deformationは,モース関数を用いてラプラス方程式を摂動する手法であり,量子力学の場の理論における超対称性とよばれる理論とモース理論との関係を追求する中で生まれたものである.

Morse Novikov理論もこれらの分野間の結びつきに関わっている.またNovikovの多重値のモース理論というアイデアもハミルトン力学系という数理物理から生まれたアイデアである.このように,申請者の本研究テーマは数学と物理学に範囲が広く及ぶ遠大なテーマである.
\textbf{\ul{そのため,この多岐にわたる分野の研究者たちとそれぞれ交流が行え,また広い知識を用いて異なる分野間を繋ぐ役目を担えるだろう申請者は,モース理論のこの周辺の分野の研究者としてふさわしい人物である}}.


\noindent
\fbox{\textbf{セミナーの主催者としての能力}}

申請者は京都大学内でオフィスアシスタントと呼ばれる計算機の管理の仕事を修士から行っている.プログラミングにもとより関心が強く,個人でも学習もしておりそのような仕事に関心があった.主な仕事内容は数学科のwebページの管理をはじめ,PCやプリンタ等の大学内で使用する機器の使用方法のマニュアルの作成などである.大学内のサーバーの構造や通信の仕組みについて日々の業務によりその知識を深めている.新型コロナウイルスはセミナーや研究集会の在り方を大幅に変化させ,オンライン化の指向性を極めて高めた.今後もオンラインで研究集会を行う流れはあると考えられるため,オフィスアシスタントで得た知識や経験は貴重なものである.\textbf{\ul{そのため申請者はオンラインで開催されたセミナーでのトラブルの対応など,研究集会の主催者として適切な能力を有している.上述の様々な分野の研究者と交流を行うという役割を,セミナーを自ら企画することで果たしていく}}.



\noindent
\fbox{\textbf{向上心}}

申請者は大学院で物理学から数学へ専攻を変えたように,幅広く数学,物理学,プログラミングの学習や研究活動に取り組んできた.このことは,幅広い分野の知識を得ようとする申請者の向上心の高さや,粘り強く研究に従事するために必要な忍耐強さを示している.
\textbf{\ul{また大学院の講義では優秀な成績を収め,授業料は全て免除され,京都大学内の理学研究科数学・数理解析専攻学生に対する支援に採用され,支援金が給付された}}.これは向上心を持って幅広く,そしてきちんと深く数学や物理学を学習してきたことを表している.



\vspace{5mm}
\noindent
\textbf{(2) 今後研究者として更なる発展のため必要と考えている要素}

理想の研究者としての在り方とそれに近づくために必要な要素について次項で述べるため,本項では研究の際の技能的な部分に関しての必要な要素について述べる.
研究を行う際に最も重要な能力は,\textbf{\ul{数学のより深い知識であり,分野を横断するような幅広い見識である}}.申請者が関心のあるモース理論はとりわけ他分野との繋がりが強いからである.
そしてその幅広い見識を手に入れることを可能にするのは,\textbf{\ul{互いに教え合う様々な分野の友人や,自身の考えをわかりやすく伝える能力}}である.


\noindent
\fbox{\textbf{知識の拡充}}

数学の当該分野についての更なる知識がまず求められる.現在は自分が関心のある論文を指導教官とのセミナーで読んで発表をしているが,今後はさらに論文の輪読会を企画していくことで,数学の知識や共同研究者をさらに増やせるだろう.



\noindent
\fbox{\textbf{幅広い交友関係}}

不慣れな分野で迷い込まないよう的確なアドバイスをくれる先達や,同じ分野の志を共にする友人が研究活動において必要である.
セミナーや研究集会に積極的に参加し発言もしっかりすることや,セミナーを主催することでそのような友人は増やせる.\textbf{\ul{そのため今後は,申請者の数学と物理学の幅広い知識を用いて,多くの分野の研究者と交流をし,分野を横断するようなセミナーを主催していく}}.ティーチングアシスタントを通して積んだ後輩の学生と交流をする経験や,オフィスアシスタントで培ったセミナーを主催する能力を通して自らセミナーを企画することを実現する.



\noindent
\fbox{\textbf{伝達力の更なる向上}}

セミナー等で自分のアイデアを円滑に伝えられることは交友関係を築くことや知識を交換する際に重要である.
セミナーでのそのようなプレゼンテーションの能力は,自らが学会やセミナーで発表することを重ねることで身についていくものである.その上で,発表中の伝わりやすい言葉や話し方,スライドの作り方について普段のセミナーから拘っていく必要がある.自身が興味深いと感じるアイデアを生き生きと伝えられるようになりたい.


%end 自己分析 ====================

\input{pieces/p04_abilities_01}

%#Split: 05_my_ambitions
%#PieceName: p05_my_ambitions
\input{pieces/p05_my_ambitions_00}
\section{目指す研究者像等}
%    <<最大 1ページ>>

% s17_my_ambitions
\noindent
\textbf{(1)目指す研究者像 {\footnotesize ※目指す研究者像に向けて身に付けるべき資質も含め記入してください。}}

%begin 目指す研究者像 ====================
申請者の理想の研究者像は\textbf{\ul{「数学,物理学の幅広い知識を生かし幾何学およびその周辺分野の発展に貢献し続ける研究者」}}である.自分が魅入られた数学,特にモース理論の最先端の研究を行いたい.それだけでなく,常に全力で好奇心を保ち研究生活を駆け抜けたい.そのために以下の能力を兼ね備えた研究者を目指す.


\noindent
\fbox{\textbf{1.深掘りする思考力}}

数学の最大の魅力は,理論が抽象化されているが故の適応範囲の広さであると申請者は考える.数学が適応範囲が広い学問であるということは,古くから科学の発展に数学が寄与し続けてきたことが示している.抽象化されたインパクトの大きい理論を構築するためには,現象の本質を抽出する能力が必要であり,\textbf{\ul{透徹した理論に多く触れることでその本質を見通す感覚は養われる}}と申請者は考えている.


\noindent
\fbox{\textbf{2.コミュニティを拡充させる能力}}

本質を抽出する能力が問題の解決の糸口をみつけたり,新たな理論を構築するアイデアを見つけてくれるならば,そのアイデアを遂行する能力も必要である.現代の複雑化した数学の問題を解決するにおいて,自身が数学に造詣が深く計算力を有していることはもちろん,他の研究者との協力が必須である.\textbf{\ul{自らが関わる分野の面白さを外部へ伝え,研究を志すものを増やし分野の参入者や共同研究者を増やす活動}}が重要である.


\noindent
\fbox{\textbf{3.挑み続ける粘り強さ}}

長い期間全力で数学の研究を続けるためには.数学の技能に加え粘り強さが重要である.数学を好きで居続ける,情熱を燃やし続けることが大切である.チャレンジングな課題を見つけ続け,さまざまな分野の新しい知識に触れそして感動をする経験を多くするべきである.そのため,数学のモース理論以外はもちろん\textbf{\ul{その他の分野の学習も欠かさず,アンテナを常に張っておくことが大切と申請者は考えている}}.

%end 目指す研究者像 ====================

\vspace{5mm}
\noindent
\textbf{(2)上記の「目指す研究者像」に向けて、特別研究員の採用期間中に行う研究活動の位置づけ}


%begin 研究活動の位置づけ ====================

本研究活動を通して上の3つの能力は自然に向上されていく.以下に理由を述べる.


\noindent\textbf{1. }80年代にNovikovにより生まれた本研究テーマは,シンプルなモース理論の拡張にもかかわらず広域な分野との繋がりがあり,また最先端の理論であるフレアーホモロジーの土台になっている.シンプルで応用が広いこの理論はモース理論の本質に深く迫る理論であり,数学のその他の分野に対しても見本になるような一般化の例に申請者は思えた.そのような理論に触れることが,申請者が求める本質を見通す能力を向上させるであろう.


\noindent\textbf{2. }ゲージ理論などの研究集会等に参加し個人の数学の技能を磨くことに加え,本研究活動の間に得られた結果を積極的に外部に発信していく.
申請者の数学と物理学の幅広い知識を用いて,多くの分野の研究者と交流をしたり分野を横断するようなセミナーを主催し
,数学の楽しみを共有できる友人を増やすことができればそれは望外の喜びである.


\noindent\textbf{3. }
\textbf{\ul{トポロジーと解析学の分野にまたがる本研究活動は,幅広い知識を活用したい申請者にとって,2つの分野を繋ぐ好奇心を強く刺激してくれる研究活動である}}.
また,本研究テーマにはさらなる応用の可能性がある.近年,パーシステントホモロジーとよばれるトポロジーを用いたデータ解析の手法が工学で活発に研究されている.プロットされたデータの点集合を半径を持たせた円にかえ,その半径を動かした時の図形のトポロジーの変化からデータの複雑さを図る手法である.この手法はモース理論,Morse Novikov理論との関係も指摘されており,それは本理論の適用範囲
がとても広いことを表している.そのためMorse Novikov理論は数学,物理学やそれ以外の分野との関係が存在し,好奇心を強く満たしてくれるような魅力的な理論である.本研究活動が幅広い見識を与えてくれ,好奇心を刺激してくれることは確実である.




%end 研究活動の位置づけ ====================

\input{pieces/p05_my_ambitions_01}

%#Split: 99_tail
\input{pieces/hook9} % pieces
\end{document}
